
\documentclass[11pt]{article}
\usepackage{amsmath}
\usepackage{amssymb}
\usepackage[doublespacing]{setspace}
\usepackage{geometry}
\usepackage{listings}

\newenvironment{enum1}
{\begin{enumerate}
  \setlength{\itemsep}{1pt}
  \setlength{\parskip}{0pt}
  \setlength{\parsep}{0pt}
}{\end{enumerate}}


\setcounter{MaxMatrixCols}{10}

\geometry{left=1in,right=1in,top=1in,bottom=1in}

\begin{document}

\title{Assignment on Logistic Regression }

\maketitle

\pagestyle{headings}

{\bf Do the following parts below. Data Analyses 2.1 (25pts), 2.3 (25pts), 2.4(35pts), Simulate and Analyze 3.2 (15pts)}

{\bf Bonus 15\% : Part 1 Problems}

\section{Problems}

\begin{enumerate}

\item 
\begin{enumerate}
\item Write the log likelihood for the logistic regression model \\ logit$($Pr$[Y|X_1=x_1,X_2=x_2])=\beta_0 + \beta_1 x_1 + \beta_2 x_2$.
\item Differentiate with respect to $\beta_0$.
\item Let $f_i$ be the linear combination $\beta_0 + \beta_1 x_{i1} + \beta_2 x_{i2}$ and $p_i = \exp[f_i]/(1+\exp[f_i])$. Interpret $p_i$.
\item At the maximum likelihoood estimate the derivative above equals zero. Equate the derivative to zero and write in terms of $p_i$. What does the sum $\sum_{i=1}^n p_i$ equal, and what does the mean $\sum_{i=1}^n p_i/n$ equal ?
\item How would you describe $\sum (y_i - p_i)^2/n$ ?
\end{enumerate}
\end{enumerate} 


\section{Data Analyses}


\subsection{Analysis of Burn Data}

\begin{enumerate}

\item Install and utilize the R library {\it aplore3}. Using the dataset {\it burn1000} develop a model for predicting death. 
\item Report the C-index.
\item Is the effect of {\it inh\_inj} on mortality modified by age?
\item Is the effect of age  on mortality modified by {\it inh\_inj}?

\end{enumerate}

\subsection{Analysis of University Admissions Data}

Read in the data using the R code read.csv("https://stats.idre.ucla.edu/stat/data/binary.csv"). The dependent variable is admit.
\begin{enumerate}

\item Report the univariable associations of GPA, GRE and Rank with Admit.
\item Develop a multivariable model.
\item Interpret the coefficients in your model.
\item Do any quadratic terms, or interactions add predictive ability?
\item Report the C-statistic.

\end{enumerate}

\subsection{Data With a Zero Cell}

Create the following dataset consisting of a dependent variable, Success, and two co-variates, Treatment and Female, using the following 3 lines of code:

\vspace{3mm}
\begin{spacing}{.25}
\begin{lstlisting}
Treatment = rep(c(0,1,0,1),each=10)
Female = rep(c(0,1),each=20)
Success = rep(rep(0:1,4), times=c(8,2,5,5,5,5,0,10))
\end{lstlisting}
\end{spacing}

\vspace{5mm}

\begin{enumerate}
\item Calculate the success frequency for the 4 combinations of Treatment and Gender. 
\item Estimate the odds ratio relating Success to Treatment.
\item Estimate the odds ratio relating Success to Gender. 
\item Include in a logistic regression the interaction of Treatment and Gender and comment on its statistical significance and coefficient.
\end{enumerate}

\subsection{Concussion Data}

Run the following code to read in and restructure a dataset that recorded concussions in college sports according to sex of athlete, sport and year. The columns in the matrix (data.frame) named Y are the number of athletes with and without concussions respectively.

\vspace{3mm}
\begin{spacing}{.25}
\begin{lstlisting}
DF <- read.delim("http://users.stat.ufl.edu/~winner/data/concussion.dat", sep="", header=FALSE)
names(DF) <- c("Sex","Sport","Year","Concussion","Count")
DF0 <- DF[DF$Concussion==0,]
DF1 <- DF[DF$Concussion==1,]
Cov <- data.frame(DF0[,1:3])
Y <- cbind(CountConc=DF1[,5], CountNoConc=DF0[,5])
\end{lstlisting}
\end{spacing}


\begin{enumerate}
\item Derive the contingency table of concussion by sex.
\item Calculate risk (frequency) of concussions by sex, and the risk ratio comparing males to females.
\item Apply Pearson's chi-square test to the contingency table.
\item Use logistic regression to test if concussions are equally likely between males and femakles.
\item Repeat the steps above substituting the variables sports for sex.
\item Run a multivariable logistic regression of concusions by sex, sports and year.
\item Report the adjusted odds ratios for sex and sports.
\item Test if there is an interaction of sex and sports.
\end{enumerate}


\section{Simulate and Analyze}

\begin{enumerate}

\item Run the code below. Then try different arguments to the function, f, e.g. try f(N0=30,N1=30, mu0=0, mu1=0.5). What is this code illustrating?

\begin{spacing}{.25}
\begin{lstlisting}
f <- function(R=500, N0=30, N1=30, mu0=0, mu1=0, sd0=1, sd1=0) {
  ptt <- plinear <- plogistic <- rep(NA, R)
  for (r in 1:R) {
    Y0 <- rnorm(n=N0, mean=mu0, sd=sd0)
    Y1 <- rnorm(n=N1, mean=mu1, sd=sd1)
    ptt[r] <-  t.test(Y0, Y1)$p.value
    Y <- c(Y0, Y1)
    X <- rep(0:1, times=c(N0,N1))
    plinear[r] <- summary(lm(Y ~ X))$coef["X",4]
    plogistic[r] <- summary(glm(X ~ Y, family=binomial))$coef["Y",4]
    }
  par(mfrow=c(1,2))
  plot(plogistic, ptt)
  plot(plogistic, plinear)
  print(table(plogistic < 0.05, ptt <0.05))
  print(summary(plogistic - ptt))
  print(table(plogistic < 0.05, plinear <0.05))
  print(summary(plogistic - plinear))
  }

f()
\end{lstlisting}
\end{spacing}

\item Explain why the estimate of the coefficient for X in the logistic regression adjusting for covariate Z1 (see below) is significantly different from zero despite the causal effect being zero?


\begin{spacing}{.25}
\begin{lstlisting}
n = 2500
Z1 = rnorm(n)
Z2 = rnorm(n)
X = 0.7*rnorm(n) + 0.7*Z2
Lin = 0*X - 0.0*Z1 + 0.5*Z2 # causal model
Y = runif(n) < 1/(1+exp(-Lin))
summary(glm(Y ~ X + Z1, family=binomial))
\end{lstlisting}
\end{spacing}

\item (a) What does the following simulated data and analysis indicate about {\it probit} regression? (b) 
Comment on the simlarities and differences the probit and logistic regressions, such as the Z values for the three covariates in the model.

\begin{spacing}{.25}
\begin{lstlisting}
beta <- c(1, -1, +2)
cutoff <- 0.5
n <- 10^4
X <- cbind(runif(n) < 0.25, runif(n) < 0.50, rnorm(n))
Y <- X %*% beta + rnorm(n)
binary.Y <- ifelse(Y < cutoff, 1, 0)

summary(X)
summary(glm(binary.Y ~ X, family=binomial(probit))
summary(glm(binary.Y ~ X, family=binomial) # for comparison with probit
\end{lstlisting}
\end{spacing}

\end{enumerate}


\end{document}
